% Options for packages loaded elsewhere
% Options for packages loaded elsewhere
\PassOptionsToPackage{unicode}{hyperref}
\PassOptionsToPackage{hyphens}{url}
\PassOptionsToPackage{dvipsnames,svgnames,x11names}{xcolor}
%
\documentclass[
  letterpaper,
  DIV=11,
  numbers=noendperiod,
  openany]{scrartcl}
\usepackage{xcolor}
\usepackage{amsmath,amssymb}
\setcounter{secnumdepth}{5}
\usepackage{iftex}
\ifPDFTeX
  \usepackage[T1]{fontenc}
  \usepackage[utf8]{inputenc}
  \usepackage{textcomp} % provide euro and other symbols
\else % if luatex or xetex
  \usepackage{unicode-math} % this also loads fontspec
  \defaultfontfeatures{Scale=MatchLowercase}
  \defaultfontfeatures[\rmfamily]{Ligatures=TeX,Scale=1}
\fi
\usepackage{lmodern}
\ifPDFTeX\else
  % xetex/luatex font selection
  \setmonofont[Scale=1.0]{Source Code Pro}
\fi
% Use upquote if available, for straight quotes in verbatim environments
\IfFileExists{upquote.sty}{\usepackage{upquote}}{}
\IfFileExists{microtype.sty}{% use microtype if available
  \usepackage[]{microtype}
  \UseMicrotypeSet[protrusion]{basicmath} % disable protrusion for tt fonts
}{}
\makeatletter
\@ifundefined{KOMAClassName}{% if non-KOMA class
  \IfFileExists{parskip.sty}{%
    \usepackage{parskip}
  }{% else
    \setlength{\parindent}{0pt}
    \setlength{\parskip}{6pt plus 2pt minus 1pt}}
}{% if KOMA class
  \KOMAoptions{parskip=half}}
\makeatother
% Make \paragraph and \subparagraph free-standing
\makeatletter
\ifx\paragraph\undefined\else
  \let\oldparagraph\paragraph
  \renewcommand{\paragraph}{
    \@ifstar
      \xxxParagraphStar
      \xxxParagraphNoStar
  }
  \newcommand{\xxxParagraphStar}[1]{\oldparagraph*{#1}\mbox{}}
  \newcommand{\xxxParagraphNoStar}[1]{\oldparagraph{#1}\mbox{}}
\fi
\ifx\subparagraph\undefined\else
  \let\oldsubparagraph\subparagraph
  \renewcommand{\subparagraph}{
    \@ifstar
      \xxxSubParagraphStar
      \xxxSubParagraphNoStar
  }
  \newcommand{\xxxSubParagraphStar}[1]{\oldsubparagraph*{#1}\mbox{}}
  \newcommand{\xxxSubParagraphNoStar}[1]{\oldsubparagraph{#1}\mbox{}}
\fi
\makeatother

\usepackage{color}
\usepackage{fancyvrb}
\newcommand{\VerbBar}{|}
\newcommand{\VERB}{\Verb[commandchars=\\\{\}]}
\DefineVerbatimEnvironment{Highlighting}{Verbatim}{commandchars=\\\{\}}
% Add ',fontsize=\small' for more characters per line
\usepackage{framed}
\definecolor{shadecolor}{RGB}{241,243,245}
\newenvironment{Shaded}{\begin{snugshade}}{\end{snugshade}}
\newcommand{\AlertTok}[1]{\textcolor[rgb]{0.68,0.00,0.00}{#1}}
\newcommand{\AnnotationTok}[1]{\textcolor[rgb]{0.37,0.37,0.37}{#1}}
\newcommand{\AttributeTok}[1]{\textcolor[rgb]{0.40,0.45,0.13}{#1}}
\newcommand{\BaseNTok}[1]{\textcolor[rgb]{0.68,0.00,0.00}{#1}}
\newcommand{\BuiltInTok}[1]{\textcolor[rgb]{0.00,0.23,0.31}{#1}}
\newcommand{\CharTok}[1]{\textcolor[rgb]{0.13,0.47,0.30}{#1}}
\newcommand{\CommentTok}[1]{\textcolor[rgb]{0.37,0.37,0.37}{#1}}
\newcommand{\CommentVarTok}[1]{\textcolor[rgb]{0.37,0.37,0.37}{\textit{#1}}}
\newcommand{\ConstantTok}[1]{\textcolor[rgb]{0.56,0.35,0.01}{#1}}
\newcommand{\ControlFlowTok}[1]{\textcolor[rgb]{0.00,0.23,0.31}{\textbf{#1}}}
\newcommand{\DataTypeTok}[1]{\textcolor[rgb]{0.68,0.00,0.00}{#1}}
\newcommand{\DecValTok}[1]{\textcolor[rgb]{0.68,0.00,0.00}{#1}}
\newcommand{\DocumentationTok}[1]{\textcolor[rgb]{0.37,0.37,0.37}{\textit{#1}}}
\newcommand{\ErrorTok}[1]{\textcolor[rgb]{0.68,0.00,0.00}{#1}}
\newcommand{\ExtensionTok}[1]{\textcolor[rgb]{0.00,0.23,0.31}{#1}}
\newcommand{\FloatTok}[1]{\textcolor[rgb]{0.68,0.00,0.00}{#1}}
\newcommand{\FunctionTok}[1]{\textcolor[rgb]{0.28,0.35,0.67}{#1}}
\newcommand{\ImportTok}[1]{\textcolor[rgb]{0.00,0.46,0.62}{#1}}
\newcommand{\InformationTok}[1]{\textcolor[rgb]{0.37,0.37,0.37}{#1}}
\newcommand{\KeywordTok}[1]{\textcolor[rgb]{0.00,0.23,0.31}{\textbf{#1}}}
\newcommand{\NormalTok}[1]{\textcolor[rgb]{0.00,0.23,0.31}{#1}}
\newcommand{\OperatorTok}[1]{\textcolor[rgb]{0.37,0.37,0.37}{#1}}
\newcommand{\OtherTok}[1]{\textcolor[rgb]{0.00,0.23,0.31}{#1}}
\newcommand{\PreprocessorTok}[1]{\textcolor[rgb]{0.68,0.00,0.00}{#1}}
\newcommand{\RegionMarkerTok}[1]{\textcolor[rgb]{0.00,0.23,0.31}{#1}}
\newcommand{\SpecialCharTok}[1]{\textcolor[rgb]{0.37,0.37,0.37}{#1}}
\newcommand{\SpecialStringTok}[1]{\textcolor[rgb]{0.13,0.47,0.30}{#1}}
\newcommand{\StringTok}[1]{\textcolor[rgb]{0.13,0.47,0.30}{#1}}
\newcommand{\VariableTok}[1]{\textcolor[rgb]{0.07,0.07,0.07}{#1}}
\newcommand{\VerbatimStringTok}[1]{\textcolor[rgb]{0.13,0.47,0.30}{#1}}
\newcommand{\WarningTok}[1]{\textcolor[rgb]{0.37,0.37,0.37}{\textit{#1}}}

\usepackage{longtable,booktabs,array}
\usepackage{calc} % for calculating minipage widths
% Correct order of tables after \paragraph or \subparagraph
\usepackage{etoolbox}
\makeatletter
\patchcmd\longtable{\par}{\if@noskipsec\mbox{}\fi\par}{}{}
\makeatother
% Allow footnotes in longtable head/foot
\IfFileExists{footnotehyper.sty}{\usepackage{footnotehyper}}{\usepackage{footnote}}
\makesavenoteenv{longtable}
\usepackage{graphicx}
\makeatletter
\newsavebox\pandoc@box
\newcommand*\pandocbounded[1]{% scales image to fit in text height/width
  \sbox\pandoc@box{#1}%
  \Gscale@div\@tempa{\textheight}{\dimexpr\ht\pandoc@box+\dp\pandoc@box\relax}%
  \Gscale@div\@tempb{\linewidth}{\wd\pandoc@box}%
  \ifdim\@tempb\p@<\@tempa\p@\let\@tempa\@tempb\fi% select the smaller of both
  \ifdim\@tempa\p@<\p@\scalebox{\@tempa}{\usebox\pandoc@box}%
  \else\usebox{\pandoc@box}%
  \fi%
}
% Set default figure placement to htbp
\def\fps@figure{htbp}
\makeatother


% definitions for citeproc citations
\NewDocumentCommand\citeproctext{}{}
\NewDocumentCommand\citeproc{mm}{%
  \begingroup\def\citeproctext{#2}\cite{#1}\endgroup}
\makeatletter
 % allow citations to break across lines
 \let\@cite@ofmt\@firstofone
 % avoid brackets around text for \cite:
 \def\@biblabel#1{}
 \def\@cite#1#2{{#1\if@tempswa , #2\fi}}
\makeatother
\newlength{\cslhangindent}
\setlength{\cslhangindent}{1.5em}
\newlength{\csllabelwidth}
\setlength{\csllabelwidth}{3em}
\newenvironment{CSLReferences}[2] % #1 hanging-indent, #2 entry-spacing
 {\begin{list}{}{%
  \setlength{\itemindent}{0pt}
  \setlength{\leftmargin}{0pt}
  \setlength{\parsep}{0pt}
  % turn on hanging indent if param 1 is 1
  \ifodd #1
   \setlength{\leftmargin}{\cslhangindent}
   \setlength{\itemindent}{-1\cslhangindent}
  \fi
  % set entry spacing
  \setlength{\itemsep}{#2\baselineskip}}}
 {\end{list}}
\usepackage{calc}
\newcommand{\CSLBlock}[1]{\hfill\break\parbox[t]{\linewidth}{\strut\ignorespaces#1\strut}}
\newcommand{\CSLLeftMargin}[1]{\parbox[t]{\csllabelwidth}{\strut#1\strut}}
\newcommand{\CSLRightInline}[1]{\parbox[t]{\linewidth - \csllabelwidth}{\strut#1\strut}}
\newcommand{\CSLIndent}[1]{\hspace{\cslhangindent}#1}



\setlength{\emergencystretch}{3em} % prevent overfull lines

\providecommand{\tightlist}{%
  \setlength{\itemsep}{0pt}\setlength{\parskip}{0pt}}



 


\addtokomafont{disposition}{\rmfamily}
\usepackage{amsmath, xparse}
\usepackage{fancyvrb, fvextra}
\usepackage{listings}
\usepackage{amsfonts}
\usepackage{amssymb}
\usepackage{graphicx}
\usepackage{tcolorbox}
\usepackage{etoolbox}
\usepackage{multicol}
\usepackage{tikz}
\usepackage[table,x11names]{xcolor}
\usepackage{bm}
\usepackage{caption}
\definecolor{cornflower}{rgb}{0.12549, 0.29020, 0.52941}
\DefineVerbatimEnvironment{Highlighting}{Verbatim}{breaklines,commandchars=\\\{\}}
\KOMAoption{captions}{tableheading}
\makeatletter
\@ifpackageloaded{caption}{}{\usepackage{caption}}
\AtBeginDocument{%
\ifdefined\contentsname
  \renewcommand*\contentsname{Table of contents}
\else
  \newcommand\contentsname{Table of contents}
\fi
\ifdefined\listfigurename
  \renewcommand*\listfigurename{List of Figures}
\else
  \newcommand\listfigurename{List of Figures}
\fi
\ifdefined\listtablename
  \renewcommand*\listtablename{List of Tables}
\else
  \newcommand\listtablename{List of Tables}
\fi
\ifdefined\figurename
  \renewcommand*\figurename{Figure}
\else
  \newcommand\figurename{Figure}
\fi
\ifdefined\tablename
  \renewcommand*\tablename{Table}
\else
  \newcommand\tablename{Table}
\fi
}
\@ifpackageloaded{float}{}{\usepackage{float}}
\floatstyle{ruled}
\@ifundefined{c@chapter}{\newfloat{codelisting}{h}{lop}}{\newfloat{codelisting}{h}{lop}[chapter]}
\floatname{codelisting}{Listing}
\newcommand*\listoflistings{\listof{codelisting}{List of Listings}}
\makeatother
\makeatletter
\makeatother
\makeatletter
\@ifpackageloaded{caption}{}{\usepackage{caption}}
\@ifpackageloaded{subcaption}{}{\usepackage{subcaption}}
\makeatother
\usepackage{bookmark}
\IfFileExists{xurl.sty}{\usepackage{xurl}}{} % add URL line breaks if available
\urlstyle{same}
\hypersetup{
  colorlinks=true,
  linkcolor={blue},
  filecolor={Maroon},
  citecolor={Blue},
  urlcolor={Blue},
  pdfcreator={LaTeX via pandoc}}


\author{}
\date{}
\begin{document}

\begin{titlepage}
  \vspace{3in}
  \newcommand{\HRule}{\rule{\linewidth}{0.5mm}}
  \begin{center}
    \textsc{\LARGE Georgia Southern University \\ \vspace{0.2cm} \Large Allen E. Paulson College of Engineering and Computing \\ Department of Computer Science}\\[0.3cm]
    \vspace{0.5cm}
    {\huge \bfseries CSCI 3432: Database Systems} \\[0.2cm]
    \HRule \\[0.4cm]
    { \huge \bfseries  PharmaCheck \\[0.05cm]
    \HRule \\[0.5cm]
    }
    
    \begin{minipage}{0.4\textwidth}
    \begin{flushleft} \Large
    Anish Goyal
    \end{flushleft}
    \end{minipage}
    ~
    \begin{minipage}{0.4\textwidth}
    \begin{flushright} \Large
    Dr. Weitian Tong \\ Associate Professor 
    \end{flushright}
    \end{minipage}\\[0.5cm]
    
    {\huge December 4, 2025}\\[0.5cm]

    \includegraphics[width=0.55\textwidth]{img/logo.png}\\
  \end{center}

  \section*{Objective}
  \noindent This project presents PharmaCheck, a web-based drug interaction checking system. I built PharmaCheck using a MySQL database backend, a Flask API server, and a modern HTML/CSS/JavaScript frontend. The system scrapes real-time drug interaction data from Drugs.com, stores it in a normalized relational database, and uses a locally deployed Ollama large language model to translate professional medical descriptions into patient-friendly language. PharmaCheck supports doctor-patient relationships, comprehensive search history tracking, and checks for drug-drug, food-lifestyle, and disease interactions.
\end{titlepage}

\clearpage

\renewcommand*\contentsname{Table of Contents}
{
\hypersetup{linkcolor=}
\setcounter{tocdepth}{4}
\tableofcontents
}

\listoffigures
\clearpage

\section{Introduction}\label{introduction}

This semester I received a cancer diagnosis. Treatment required managing
multiple medications simultaneously. I witnessed firsthand how doctors
track complex drug interactions using internal hospital systems.
Patients like me often lack visibility into potential medication
conflicts. This experience inspired me to build an accessible system for
drug interaction awareness.

Drug interactions cause over 100,000 hospitalizations annually in the
United States. According to recent statistics, 82\% of Americans take at
least one medication. 29\% take five or more medications daily. Drug
interactions rank as the fourth leading cause of death in the US.
Healthcare providers use proprietary internal systems that patients
cannot access. Existing public tools lack transparency and real-time
accuracy.

PharmaCheck is a database-managed web application that integrates
artificial intelligence and advanced indexing to identify adversarial
drug interactions in patients. I built this platform to address the gap
between professional medical systems and patient needs. The system
provides real-time drug interaction checking, food and lifestyle
interaction warnings, disease interaction alerts, and AI-powered
translation of medical terminology.

The main components of PharmaCheck include a MySQL relational database
storing drugs, conditions, and interaction data. A Flask backend serves
a RESTful API for all operations. The frontend uses vanilla HTML, CSS,
and JavaScript without heavy framework overhead. A web scraping system
fetches real-time data from Drugs.com. An Ollama large language model
translates professional descriptions into patient-friendly language. A
doctor-patient relationship system enables medical oversight. A
comprehensive search history feature allows users to revisit previous
queries.

\begin{figure}[H]

\centering{

\includegraphics[width=0.7\linewidth,height=\textheight,keepaspectratio]{./img/tech_stack.png}

}

\caption{\label{fig-tech-stack}Technology stack showing MySQL, Flask,
and frontend components}

\end{figure}%

Figure~\ref{fig-tech-stack} shows the technology stack I chose for
PharmaCheck. MySQL provides reliable relational storage. Flask offers a
lightweight Python web framework. The frontend prioritizes simplicity
and performance.

\clearpage

\section{Database Details}\label{database-details}

I designed the database schema through careful entity-relationship
modeling. The design process began with identifying core entities:
users, drugs, conditions, and interactions. I mapped relationships
between these entities and planned for real-time data acquisition from
external sources.

\begin{figure}[H]

\centering{

\includegraphics[width=0.9\linewidth,height=\textheight,keepaspectratio]{./img/er_diagram.png}

}

\caption{\label{fig-er-diagram}Entity-Relationship diagram for
PharmaCheck}

\end{figure}%

Figure~\ref{fig-er-diagram} shows the ER model I created during the
initial design phase. The model captures users with distinct roles,
drugs with their properties, conditions they treat, and various types of
interactions between drugs.

The final schema contains nine tables. The User table stores
authentication credentials and role information. Each user has a unique
identifier, username, password hash, email, and role designation as
either PATIENT or DOCTOR. The Drug table contains medication information
including name, generic name, description, URL reference, and an
optional foreign key to the Condition table. The Condition table stores
medical conditions with names, descriptions, and URLs. The Interaction
table captures drug-drug interaction details including severity level,
professional description, patient description, and cached AI-generated
description.

I created a junction table called Drug\_Interaction to handle the
many-to-many relationship between drugs and interactions. This table
contains foreign keys to both the Drug and Interaction tables along with
the name of the interacting drug. The FoodInteraction table stores food
and lifestyle interactions with severity, hazard level, plausibility
rating, and descriptions. The DiseaseInteraction table follows a similar
structure for disease-related interactions. The SearchHistory table logs
all user searches with timestamps, query text, search type, and full
JSON results for restoration. The Doctor\_Patient table manages the
many-to-many relationship between doctors and their patients.

The functional dependencies in each table follow Third Normal Form
requirements. For the User table: user\_id determines username,
password\_hash, email, role, created\_at, and updated\_at. For the Drug
table: drug\_id determines name, generic\_name, description, url,
condition\_id, created\_at, and updated\_at. For the Interaction table:
interaction\_id determines severity, professional\_description,
patient\_description, ai\_description, url, created\_at, and
updated\_at. For SearchHistory: search\_id determines user\_id, query,
search\_type, search\_data, and created\_at.

All tables satisfy Third Normal Form. No transitive dependencies exist
because all non-key attributes depend directly on the primary key. I
avoided storing derived data. Each attribute represents a single fact
about the entity identified by the primary key.

I implemented several constraint types to maintain data integrity.
Primary keys use auto-incrementing integers for efficient indexing.
Foreign keys include CASCADE rules for both UPDATE and DELETE
operations. When a user deletes their account, all related search
history entries delete automatically. The UNIQUE constraint applies to
username and email fields in the User table to prevent duplicates. NOT
NULL constraints ensure required fields always contain values. ENUM
types restrict the role field to PATIENT or DOCTOR values and the
severity field to Major, Moderate, Minor, or Unknown values.

The following SQL shows the User table definition:

\begin{Shaded}
\begin{Highlighting}[numbers=left,,]
\KeywordTok{CREATE} \KeywordTok{TABLE} \FunctionTok{User}\NormalTok{ (}
\NormalTok{    user\_id }\DataTypeTok{INT} \KeywordTok{PRIMARY} \KeywordTok{KEY}\NormalTok{ AUTO\_INCREMENT,}
\NormalTok{    username }\DataTypeTok{VARCHAR}\NormalTok{(}\DecValTok{64}\NormalTok{) }\KeywordTok{NOT} \KeywordTok{NULL} \KeywordTok{UNIQUE}\NormalTok{,}
\NormalTok{    password\_hash }\DataTypeTok{CHAR}\NormalTok{(}\DecValTok{60}\NormalTok{) }\KeywordTok{NOT} \KeywordTok{NULL}\NormalTok{,}
\NormalTok{    email }\DataTypeTok{VARCHAR}\NormalTok{(}\DecValTok{255}\NormalTok{) }\KeywordTok{NOT} \KeywordTok{NULL} \KeywordTok{UNIQUE}\NormalTok{,}
    \KeywordTok{role}\NormalTok{ ENUM(}\StringTok{\textquotesingle{}PATIENT\textquotesingle{}}\NormalTok{, }\StringTok{\textquotesingle{}DOCTOR\textquotesingle{}}\NormalTok{) }\KeywordTok{NOT} \KeywordTok{NULL}\NormalTok{,}
\NormalTok{    created\_at DATETIME }\KeywordTok{NOT} \KeywordTok{NULL} \KeywordTok{DEFAULT} \FunctionTok{CURRENT\_TIMESTAMP}\NormalTok{,}
\NormalTok{    updated\_at DATETIME }\KeywordTok{NOT} \KeywordTok{NULL} \KeywordTok{DEFAULT} \FunctionTok{CURRENT\_TIMESTAMP} 
        \KeywordTok{ON} \KeywordTok{UPDATE} \FunctionTok{CURRENT\_TIMESTAMP}
\NormalTok{);}
\end{Highlighting}
\end{Shaded}

The SearchHistory table definition includes the search\_data column for
storing complete JSON results:

\begin{Shaded}
\begin{Highlighting}[numbers=left,,]
\KeywordTok{CREATE} \KeywordTok{TABLE}\NormalTok{ SearchHistory (}
\NormalTok{    search\_id BIGINT }\KeywordTok{PRIMARY} \KeywordTok{KEY}\NormalTok{ AUTO\_INCREMENT,}
\NormalTok{    user\_id }\DataTypeTok{INT} \KeywordTok{NOT} \KeywordTok{NULL}\NormalTok{,}
    \KeywordTok{query}\NormalTok{ TEXT }\KeywordTok{NOT} \KeywordTok{NULL}\NormalTok{,}
\NormalTok{    search\_type ENUM(}\StringTok{\textquotesingle{}DRUG\textquotesingle{}}\NormalTok{, }\StringTok{\textquotesingle{}CONDITION\textquotesingle{}}\NormalTok{, }\StringTok{\textquotesingle{}INTERACTION\textquotesingle{}}\NormalTok{, }
        \StringTok{\textquotesingle{}FOOD\_INTERACTION\textquotesingle{}}\NormalTok{, }\StringTok{\textquotesingle{}DISEASE\_INTERACTION\textquotesingle{}}\NormalTok{) }\KeywordTok{DEFAULT} \StringTok{\textquotesingle{}DRUG\textquotesingle{}}\NormalTok{,}
\NormalTok{    search\_data TEXT,}
\NormalTok{    created\_at DATETIME }\KeywordTok{NOT} \KeywordTok{NULL} \KeywordTok{DEFAULT} \FunctionTok{CURRENT\_TIMESTAMP}\NormalTok{,}
    \KeywordTok{FOREIGN} \KeywordTok{KEY}\NormalTok{ (user\_id) }\KeywordTok{REFERENCES} \FunctionTok{User}\NormalTok{(user\_id)}
        \KeywordTok{ON} \KeywordTok{UPDATE} \KeywordTok{CASCADE} \KeywordTok{ON} \KeywordTok{DELETE} \KeywordTok{CASCADE}\NormalTok{,}
    \KeywordTok{INDEX}\NormalTok{ idx\_search\_user (user\_id),}
    \KeywordTok{INDEX}\NormalTok{ idx\_search\_created (created\_at)}
\NormalTok{);}
\end{Highlighting}
\end{Shaded}

I added indexes on frequently queried columns. The drug name and generic
name columns have indexes for fast autocomplete searches. The search
history table has indexes on user\_id and created\_at for efficient
retrieval of recent searches. Full-text indexes on drug and condition
names support future advanced search features.

\clearpage

\section{Functionality Details}\label{functionality-details}

I implemented both basic and advanced functions in PharmaCheck. The
basic functions handle user authentication, drug searching, and simple
interaction checks. The advanced functions provide multi-drug analysis,
AI translation, and doctor-patient management.

User registration accepts a username, email, password, and role
selection. Patients can optionally select a doctor during registration
for immediate oversight assignment. The system hashes passwords using
bcrypt before storage. Upon successful registration, the API returns a
JWT token for subsequent authenticated requests. Login accepts username
or email with password verification. The JWT token expires after a
configurable duration, defaulting to one hour.

Drug search uses database-backed autocomplete. When you type at least
two characters, the system queries the Drug table using prefix matching.
The database contains 15,775 drugs imported from Drugs.com. If the
database returns no matches, the system falls back to a JSON file
search. Results return drug names, URLs, and generic names when
available.

The multi-drug interaction checker accepts up to five drugs
simultaneously. For each drug in the list, the system fetches
interactions from Drugs.com and checks if other drugs in your list
appear in the interaction results. The checker uses both exact string
matching and fuzzy matching with Levenshtein distance to catch
variations in drug naming. The following pseudocode describes the core
logic:

\begin{Shaded}
\begin{Highlighting}[numbers=left,,]
\ControlFlowTok{for}\NormalTok{ each drug }\KeywordTok{in}\NormalTok{ user\_drug\_list:}
\NormalTok{    interactions }\OperatorTok{=}\NormalTok{ fetch\_interactions(drug)}
    \ControlFlowTok{for}\NormalTok{ each interaction }\KeywordTok{in}\NormalTok{ interactions:}
        \ControlFlowTok{for}\NormalTok{ each other\_drug }\KeywordTok{in}\NormalTok{ user\_drug\_list:}
            \ControlFlowTok{if}\NormalTok{ other\_drug }\OperatorTok{!=}\NormalTok{ drug:}
                \ControlFlowTok{if}\NormalTok{ other\_drug }\KeywordTok{in}\NormalTok{ interaction.name }\KeywordTok{or} 
\NormalTok{                   is\_similar(other\_drug, interaction.name):}
\NormalTok{                    add\_to\_results(drug, other\_drug, interaction)}
\end{Highlighting}
\end{Shaded}

Food and lifestyle interaction checking fetches data from a separate
Drugs.com endpoint. The scraper parses HTML to extract interaction
names, severity levels, hazard ratings, plausibility scores, and
detailed descriptions. Disease interaction checking works similarly,
extracting applicable conditions and clinical recommendations.

The AI translation feature uses a locally deployed Ollama instance
running the Llama 3.2 model. When you request a translation, the system
sends the professional description to Ollama with a prompt instructing
it to act as a clinical physician translating for patient comprehension.
The translated text caches in the database to avoid repeated API calls.
The translation function appears below:

\begin{Shaded}
\begin{Highlighting}[numbers=left,,]
\KeywordTok{def}\NormalTok{ translate\_professional\_to\_consumer(professional\_description):}
\NormalTok{    prompt }\OperatorTok{=} \SpecialStringTok{f"""Pretend you are a clinical physician. }
\SpecialStringTok{    Translate the following professional drug interaction }
\SpecialStringTok{    description into a more consumer{-}friendly description. }
\SpecialStringTok{    Write the consumer{-}friendly description only:}
\SpecialStringTok{    }
\SpecialStringTok{    }\SpecialCharTok{\{}\NormalTok{professional\_description}\SpecialCharTok{\}}\SpecialStringTok{"""}
    
\NormalTok{    response }\OperatorTok{=}\NormalTok{ requests.post(}
        \SpecialStringTok{f"}\SpecialCharTok{\{}\NormalTok{OLLAMA\_BASE\_URL}\SpecialCharTok{\}}\SpecialStringTok{/api/generate"}\NormalTok{,}
\NormalTok{        json}\OperatorTok{=}\NormalTok{\{}\StringTok{"model"}\NormalTok{: }\StringTok{"llama3.2:3b"}\NormalTok{, }
              \StringTok{"prompt"}\NormalTok{: prompt, }
              \StringTok{"stream"}\NormalTok{: }\VariableTok{False}\NormalTok{\},}
\NormalTok{        timeout}\OperatorTok{=}\DecValTok{60}
\NormalTok{    )}
    \ControlFlowTok{return}\NormalTok{ response.json().get(}\StringTok{"response"}\NormalTok{, }\StringTok{""}\NormalTok{)}
\end{Highlighting}
\end{Shaded}

The doctor-patient relationship system allows patients to request
oversight from registered doctors. Doctors can view all assigned
patients and access their search histories. This feature enables medical
professionals to monitor what drug combinations their patients
investigate. Patients retain control and can remove doctor assignments
at any time.

Search history tracking stores complete interaction results as JSON.
When you click a previous search, the system restores the full results
without re-scraping. This improves performance and provides consistent
historical data even if Drugs.com content changes. The search type field
distinguishes between drug searches, condition searches, drug-drug
interactions, food interactions, and disease interactions.

\clearpage

\section{Implementation Details}\label{implementation-details}

I chose Python 3 with Flask for the backend because of its simplicity
and extensive library ecosystem. MySQL serves as the primary database
due to its reliability and wide support. The frontend uses vanilla HTML,
CSS, and JavaScript to minimize dependencies and maximize performance.

\begin{figure}[H]

\centering{

\includegraphics[width=0.85\linewidth,height=\textheight,keepaspectratio]{./img/welcome_page.png}

}

\caption{\label{fig-welcome}Welcome page showing the PharmaCheck landing
interface}

\end{figure}%

Figure~\ref{fig-welcome} shows the welcome page that greets users. The
design emphasizes clarity and medical professionalism. Navigation links
direct users to registration, login, or the main dashboard.

\begin{figure}[H]

\centering{

\includegraphics[width=0.85\linewidth,height=\textheight,keepaspectratio]{./img/dashboard.png}

}

\caption{\label{fig-dashboard}Dashboard interface with interaction
checking options}

\end{figure}%

The dashboard in Figure~\ref{fig-dashboard} provides access to all
system features. Users can check drug interactions, view food and
lifestyle warnings, explore disease interactions, and review their
search history.

The Flask application structure follows RESTful principles.
Authentication endpoints handle registration and login at /auth/register
and /auth/login. Drug search endpoints provide autocomplete at
/drugs/autocomplete. Interaction checking occurs at
/check\_drug\_interactions for multi-drug analysis. The API returns JSON
responses for all endpoints.

SQLAlchemy provides the object-relational mapping layer. I configured
connection pooling with pool\_size=10 and max\_overflow=20 to handle
concurrent requests efficiently. Scoped sessions ensure thread safety.
The teardown\_appcontext decorator automatically closes sessions after
each request. The database URL reads from environment variables for
deployment flexibility.

\begin{figure}[H]

\centering{

\includegraphics[width=0.4\linewidth,height=\textheight,keepaspectratio]{./img/scraping_workflow.png}

}

\caption{\label{fig-scraping}Web scraping workflow from Drugs.com to
database}

\end{figure}%

Figure~\ref{fig-scraping} illustrates the web scraping workflow. The
DrugInteractionChecker class manages all scraping operations. When you
request interactions for a drug, the system first checks the database
cache. If cached data exists and use\_cache=True, the system returns
stored results immediately. Otherwise, the scraper fetches fresh data
from Drugs.com, parses the HTML using BeautifulSoup, extracts
interaction details, stores them in MySQL, and returns the results.

Brand name to generic name resolution handles cases where drugs have
multiple names. Prozac and fluoxetine refer to the same medication but
use different URL patterns on Drugs.com. The scraper checks the drug's
main page to find the generic name, then uses that for interaction
lookups. This ensures complete interaction data regardless of which name
you enter.

\begin{figure}[H]

\centering{

\includegraphics[width=0.35\linewidth,height=\textheight,keepaspectratio]{./img/drug_checker.png}

}

\caption{\label{fig-checker}Drug interaction checker showing
severity-coded results}

\end{figure}%

Figure~\ref{fig-checker} displays the interaction checker interface.
Results appear color-coded by severity. Major interactions show in red.
Moderate interactions appear in yellow. Minor interactions display in
green. Each result expands to show full professional and
patient-friendly descriptions.

Authentication uses bcrypt for password hashing with automatic salt
generation. JWT tokens encode user\_id and role claims. The
login\_required decorator validates tokens on protected routes. The
role\_required decorator restricts certain endpoints to doctors only.

\begin{figure}[H]

\centering{

\includegraphics[width=0.85\linewidth,height=\textheight,keepaspectratio]{./img/my_patients.png}

}

\caption{\label{fig-patients}Doctor's patient management view}

\end{figure}%

Figure~\ref{fig-patients} shows the doctor's view of assigned patients.
Doctors see patient usernames and recent search activity. Clicking a
patient reveals their complete search history. This enables doctors to
monitor patient medication research and provide informed guidance.

Initial data loading imports 2,123 conditions and 15,773 drugs from JSON
files. The import script processes records in batches of 100 to prevent
memory issues. Progress feedback prints every 100 records. The entire
import completes in approximately two minutes on standard hardware.

The frontend communicates with the backend through fetch API calls. All
requests include the JWT token in the Authorization header. Error
responses display user-friendly messages. Loading indicators show during
long operations like web scraping or AI translation.

\clearpage

\section{Experiences}\label{experiences}

I encountered several challenges during development. The HTML structure
on Drugs.com changed multiple times during the project. My initial
selectors stopped working when the site updated its CSS classes. I
solved this by implementing multiple fallback selectors in the parsing
code. The scraper tries several patterns until one succeeds. This
approach provides resilience against future site changes.

Brand name resolution proved more complex than expected. Searching for
``Valium'' required finding ``diazepam'' for the interaction lookup. The
drugs use entirely different URL paths. I implemented a lookup function
that visits the brand name page, extracts the generic name from the
HTML, and uses that for subsequent requests. This adds an extra HTTP
request but ensures accurate results.

Performance became a concern when initial scraping took over two minutes
per multi-drug check. Users experienced unacceptable wait times. I
implemented intelligent caching that stores scraped interactions in
MySQL. Subsequent requests for the same drug return cached data
instantly. The use\_cache parameter allows forcing fresh data when
needed. This reduced typical response times to under one second for
cached drugs.

Schema evolution mid-development taught me the importance of migration
planning. I initially designed only drug-drug interactions. User
feedback requested food and disease interactions. Adding the
FoodInteraction and DiseaseInteraction tables required careful
migration. I created a separate migration SQL file to add the new
columns and modify ENUM values. Future projects will include schema
versioning from the start.

Working with real-world web data reinforced the need for defensive
programming. HTML parsing can fail in countless ways. Missing elements,
changed class names, and unexpected content all cause errors. I wrapped
parsing operations in try-except blocks and provided sensible defaults.
The system degrades gracefully rather than crashing on unexpected input.

The doctor-patient feature design went through multiple iterations. My
original design had doctors adding patients to their list. User testing
revealed this felt intrusive. I reversed the flow so patients request
doctor oversight. Patients retain control while doctors can still
monitor those who opt in. This matches the real-world dynamic better.

Several lessons emerged from this project. Real databases require
planning for change. Schema evolution happens in every long-lived
application. Building migration support early saves time later. Web
scraping needs resilience and fallback strategies. External data sources
change without notice. Caching dramatically improves user experience for
repeated queries.

Future development could extend PharmaCheck in several directions. A
background job queue using Celery would allow asynchronous scraping.
Users could submit requests and receive notifications when results are
ready. Elasticsearch integration would enable full-text search across
drug descriptions. A React Native mobile application would reach users
on smartphones. PDF report generation would let users share interaction
reports with healthcare providers. Integration with pharmacy systems
could pre-populate current medications. An analytics dashboard would
help doctors track patient adherence patterns.

The project demonstrates practical application of database concepts from
this course. Normalization ensures data integrity. Indexing enables fast
searches. Foreign key constraints maintain referential integrity.
Transaction management prevents data corruption. These concepts moved
from theoretical to practical through hands-on implementation.

\clearpage

\section*{References}\label{references}
\addcontentsline{toc}{section}{References}

\phantomsection\label{refs}
\begin{CSLReferences}{0}{0}
\bibitem[\citeproctext]{ref-Ramakrishnan_Gehrke_2003}
\CSLLeftMargin{{[}1{]} }%
\CSLRightInline{R. Ramakrishnan and J. Gehrke, \emph{Database management
systems}, 3rd Edition. McGraw-Hill, 2003.}

\bibitem[\citeproctext]{ref-DrugsComData}
\CSLLeftMargin{{[}2{]} }%
\CSLRightInline{Drugs.com, {``Drugs.com drug interaction checker.''}
\url{https://www.drugs.com/drug_interactions.html}, 2024.}

\bibitem[\citeproctext]{ref-Ollama}
\CSLLeftMargin{{[}3{]} }%
\CSLRightInline{Ollama, {``Ollama: Run large language models locally.''}
\url{https://ollama.ai}, 2024.}

\bibitem[\citeproctext]{ref-Flask}
\CSLLeftMargin{{[}4{]} }%
\CSLRightInline{Pallets Projects, {``Flask: A python microframework.''}
\url{https://flask.palletsprojects.com}, 2024.}

\bibitem[\citeproctext]{ref-SQLAlchemy}
\CSLLeftMargin{{[}5{]} }%
\CSLRightInline{SQLAlchemy, {``SQLAlchemy: The python SQL toolkit and
ORM.''} \url{https://www.sqlalchemy.org}, 2024.}

\end{CSLReferences}




\end{document}
