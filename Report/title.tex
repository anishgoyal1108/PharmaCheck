\begin{titlepage}
  \vspace{3in}
  \newcommand{\HRule}{\rule{\linewidth}{0.5mm}}
  \begin{center}
    \textsc{\LARGE Georgia Southern University \\ \vspace{0.2cm} \Large Allen E. Paulson College of Engineering and Computing \\ Department of Computer Science}\\[0.3cm]
    \vspace{0.5cm}
    {\huge \bfseries CSCI 3432: Database Systems} \\[0.2cm]
    \HRule \\[0.4cm]
    { \huge \bfseries  PharmaCheck \\[0.05cm]
    \HRule \\[0.5cm]
    }
    
    \begin{minipage}{0.4\textwidth}
    \begin{flushleft} \Large
    Anish Goyal
    \end{flushleft}
    \end{minipage}
    ~
    \begin{minipage}{0.4\textwidth}
    \begin{flushright} \Large
    Dr. Weitian Tong \\ Associate Professor 
    \end{flushright}
    \end{minipage}\\[0.5cm]
    
    {\huge December 4, 2025}\\[0.5cm]

    \includegraphics[width=0.55\textwidth]{img/logo.png}\\
  \end{center}

  \section*{Objective}
  \noindent This project presents PharmaCheck, a web-based drug interaction checking system. I built PharmaCheck using a MySQL database backend, a Flask API server, and a modern HTML/CSS/JavaScript frontend. The system scrapes real-time drug interaction data from Drugs.com, stores it in a normalized relational database, and uses a locally deployed Ollama large language model to translate professional medical descriptions into patient-friendly language. PharmaCheck supports doctor-patient relationships, comprehensive search history tracking, and checks for drug-drug, food-lifestyle, and disease interactions.
\end{titlepage}
