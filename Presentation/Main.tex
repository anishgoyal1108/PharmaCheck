\documentclass[t,aspectratio=169,xcolor=dvipsnames]{beamer}
\usefonttheme{professionalfonts}
\usetheme{SimplePlusAIC}
\usepackage{graphicx} % Allows including images
\usepackage{booktabs} % Allows the use of \toprule, \midrule and  \bottomrule in tables
% \usepackage{svg} % Removed - using PDF for logo instead
\usepackage{tikz}
\usepackage{makecell}
\newcommand*{\defeq}{\stackrel{\text{def}}{=}}
\usepackage{setspace}
\usepackage[T1]{fontenc}
\usepackage{helvet}
\usepackage{textgreek}
\usepackage{amsmath}
\usepackage{ragged2e}
\usepackage{listings}
\usepackage{xfrac}
\usepackage[nice]{nicefrac}
\usepackage[loose]{units}
\usepackage{braket}
\usepackage{physics}
\usepackage{textcomp} % Replacing gensymb for symbols like \micro and \perthousand
\usepackage[cm]{sfmath}
\usepackage{bm}
\usepackage{verbatim}
\usepackage{fancyvrb}
\usepackage{hyperref}
\usepackage[style=ieee]{citation-style-language}
\addbibresource{references.bib}
\setbeamercovered{transparent}
\beamerdefaultoverlayspecification{<+->}
\setbeamertemplate{blocks}[rounded][shadow] % block options
\usepackage[svgnames,table]{xcolor}
\usepackage{xcolor}

\definecolor{codegreen}{rgb}{0,0.6,0}
\definecolor{codegray}{rgb}{0.5,0.5,0.5}
\definecolor{codepurple}{rgb}{0.58,0,0.82}
\definecolor{backcolour}{rgb}{0.95,0.95,0.92}

\lstdefinestyle{mystyle}{
    backgroundcolor=\color{backcolour},   
    commentstyle=\color{codegreen},
    keywordstyle=\color{magenta},
    numberstyle=\tiny\color{codegray},
    stringstyle=\color{codepurple},
    basicstyle=\ttfamily\footnotesize,
    breakatwhitespace=false,         
    breaklines=true,                 
    captionpos=b,                    
    keepspaces=true,                 
    numbers=left,                    
    numbersep=5pt,                  
    showspaces=false,                
    showstringspaces=false,
    showtabs=false,                  
    tabsize=2
}

\lstset{style=mystyle}

\arrayrulecolor{black}
\setlength{\arrayrulewidth}{0.20mm}
\renewcommand{\arraystretch}{1.40}  % stretch tables row size
\tracinglostchars=0

%================================================
%	TITLE PAGE
%================================================
\title[PharmaCheck]{PharmaCheck} 
\subtitle{Drug Interaction Database System}

\author[Goyal]{\texorpdfstring{Anish Goyal}{Anish Goyal}}
\institute[GSU]{Department of Computer Science \\ Georgia Southern University \\ Statesboro, GA \\ }

\date{\textcolor{nyublue}{\today}}

%================================================
%	BEGIN DOCUMENT 
%================================================
\begin{document}
\nocite{*}

%------------------------------------------------
%	TITLE SLIDE
%------------------------------------------------
\settitlepagebackground
\begin{frame}[plain]
    \titlepage
\end{frame}
\resetbackground

%------------------------------------------------
%	OUTLINE SLIDES (2 slides)
%------------------------------------------------
{
\beamerdefaultoverlayspecification{}
\begin{frame}[plain]
    \frametitle{Outline (1/2)}
    \begin{columns}[t]
        \begin{column}{.5\textwidth}
            \begin{spacing}{1.4}
                \tableofcontents[sections={1-2}]
            \end{spacing}
        \end{column}
        \begin{column}{.5\textwidth}
            \begin{spacing}{1.4}
                \tableofcontents[sections={3-4}]
            \end{spacing}
        \end{column}
    \end{columns}
\end{frame}

\begin{frame}[plain]
    \frametitle{Outline (2/2)}
    \begin{columns}[t]
        \begin{column}{.5\textwidth}
            \begin{spacing}{1.4}
                \tableofcontents[sections={5-6}]
            \end{spacing}
        \end{column}
        \begin{column}{.5\textwidth}
            \begin{spacing}{1.4}
                \tableofcontents[sections={7-9}]
            \end{spacing}
        \end{column}
    \end{columns}
\end{frame}
}

%================================================
%	SECTION 1: INTRODUCTION
%================================================
\setsectioncontent{Personal motivation, the need for drug interaction systems, and project overview.}
\section{Introduction}

\subsection{Personal Motivation}
\begin{frame}
    \frametitle{Personal Motivation}
    \framesubtitle{Why this project matters to me}
    
    \begin{itemize}
        \item This semester, I received a cancer diagnosis
        \item Treatment required managing multiple medications simultaneously
        \item Witnessed firsthand how doctors track complex drug interactions internally
        \item Realized patients often lack visibility into potential medication conflicts
        \item Inspired to build an accessible system for drug interaction awareness
    \end{itemize}
\end{frame}

\subsection{The Need}
\begin{frame}
    \frametitle{The Need}
    \framesubtitle{Why drug interaction systems matter}
    
    \begin{columns}
        \begin{column}{0.55\textwidth}
            \begin{itemize}
                \item Drug interactions cause 100,000+ hospitalizations annually in the US
                \item Patients often take 5+ medications simultaneously
                \item Healthcare providers use internal systems not accessible to patients
                \item Existing public tools lack transparency and real-time accuracy
            \end{itemize}
        \end{column}
        \begin{column}{0.45\textwidth}
            % FIGURE: Add statistics graphic here
            \begin{block}{Key Statistics}
                \begin{itemize}
                    \item 82\% of Americans take at least 1 medication
                    \item 29\% take 5+ medications
                    \item Drug interactions are the 4th leading cause of death
                \end{itemize}
            \end{block}
        \end{column}
    \end{columns}
\end{frame}

\subsection{Project Overview}
\begin{frame}
    \frametitle{Project Overview}
    \framesubtitle{What is PharmaCheck?}
    
    \begin{columns}
        \begin{column}{0.65\textwidth}
            \begin{itemize}
                \item \textbf{PharmaCheck}: A web-based drug interaction checking platform
                \item Built with MySQL database, Flask backend, and modern HTML/CSS/JS frontend
                \item Real-time web scraping from Drugs.com for up-to-date information
                \item AI-powered translation of professional descriptions to patient-friendly language
                \item Doctor-patient relationship management for oversight
                \item Comprehensive search history tracking
            \end{itemize}
        \end{column}
        \begin{column}{0.35\textwidth}
            \centering
            \includegraphics[width=0.9\textwidth]{../Logo.pdf}
        \end{column}
    \end{columns}
    
    \vspace{0.2cm}
    \centering
    \includegraphics[width=0.35\textwidth]{figures/tech_stack.png}
\end{frame}

%================================================
%	SECTION 2: STAGE 2 - PROJECT PROPOSAL
%================================================
\setsectioncontent{Initial planning, ER diagram design, and Beta I conception.}
\section{Stage 2: Project Proposal}

\subsection{Beta I Conception}
\begin{frame}
    \frametitle{Beta I Conception}
    \framesubtitle{Initial planning and entity identification}
    
    \begin{columns}
        \begin{column}{0.5\textwidth}
            \begin{itemize}
                \item Designed Entity-Relationship diagram
                \item Identified core entities:
                \begin{itemize}
                    \item User (Patient/Doctor)
                    \item Drug
                    \item Condition
                    \item Interaction
                    \item SearchHistory
                \end{itemize}
                \item Mapped relationships between entities
                \item Planned for real-time data acquisition
            \end{itemize}
        \end{column}
        \begin{column}{0.5\textwidth}
            \centering
            \includegraphics[width=0.85\textwidth]{figures/er_diagram.png}
        \end{column}
    \end{columns}
\end{frame}

\subsection{Database Design Goals}
\begin{frame}
    \frametitle{Database Design Goals}
    \framesubtitle{Requirements for the system}
    
    \begin{itemize}
        \item Support user accounts with distinct roles (Patient, Doctor)
        \item Track drug interactions with severity levels (Major, Moderate, Minor)
        \item Enable comprehensive search history for audit and review
        \item Support doctor-patient assignments for medical oversight
        \item Store both professional and patient-friendly descriptions
        \item Cache AI-generated translations for performance
    \end{itemize}
\end{frame}

%================================================
%	SECTION 3: DATABASE DESIGN
%================================================
\setsectioncontent{Relational schema, normalization, and key table structures.}
\section{Database Design}

\subsection{Relational Schema}
\begin{frame}
    \frametitle{Relational Schema}
    \framesubtitle{Eight core tables}
    
    \begin{columns}
        \begin{column}{0.5\textwidth}
            \begin{itemize}
                \item \textbf{User} -- Authentication \& roles
                \item \textbf{Drug} -- Medication information
                \item \textbf{Condition} -- Medical conditions
                \item \textbf{Interaction} -- Drug-drug interactions
            \end{itemize}
        \end{column}
        \begin{column}{0.5\textwidth}
            \begin{itemize}
                \item \textbf{Drug\_Interaction} -- Junction table
                \item \textbf{FoodInteraction} -- Food/lifestyle
                \item \textbf{DiseaseInteraction} -- Disease interactions
                \item \textbf{SearchHistory} -- User searches
                \item \textbf{Doctor\_Patient} -- Assignments
            \end{itemize}
        \end{column}
    \end{columns}
    
\end{frame}

\subsection{Normalization}
\begin{frame}
    \frametitle{Normalization}
    \framesubtitle{Third Normal Form (3NF) compliance}
    
    \begin{itemize}
        \item All tables satisfy Third Normal Form (3NF)
        \item \textbf{User}: user\_id $\rightarrow$ username, password\_hash, email, role
        \item \textbf{Drug}: drug\_id $\rightarrow$ name, generic\_name, description, condition\_id
        \item \textbf{Interaction}: interaction\_id $\rightarrow$ severity, professional\_description, patient\_description
        \item No transitive dependencies exist
        \item Foreign keys ensure referential integrity with CASCADE operations
    \end{itemize}
\end{frame}

\subsection{Key Tables}
\begin{frame}[fragile]
    \frametitle{Key Tables Deep Dive}
    \framesubtitle{User and SearchHistory tables}
    
    \begin{columns}
        \begin{column}{0.5\textwidth}
            \textbf{User Table}
            \begin{lstlisting}[language=SQL, basicstyle=\tiny\ttfamily]
CREATE TABLE User (
  user_id INT PRIMARY KEY 
    AUTO_INCREMENT,
  username VARCHAR(64) 
    NOT NULL UNIQUE,
  password_hash CHAR(60) 
    NOT NULL,
  email VARCHAR(255) 
    NOT NULL UNIQUE,
  role ENUM('PATIENT', 
    'DOCTOR') NOT NULL
);
            \end{lstlisting}
        \end{column}
        \begin{column}{0.5\textwidth}
            \textbf{SearchHistory Table}
            \begin{lstlisting}[language=SQL, basicstyle=\tiny\ttfamily]
CREATE TABLE SearchHistory (
  search_id BIGINT PRIMARY KEY 
    AUTO_INCREMENT,
  user_id INT NOT NULL,
  query TEXT NOT NULL,
  search_type ENUM('DRUG', 
    'CONDITION', 
    'INTERACTION'),
  search_data TEXT,
  created_at DATETIME 
    DEFAULT CURRENT_TIMESTAMP,
  FOREIGN KEY (user_id) 
    REFERENCES User(user_id)
);
            \end{lstlisting}
        \end{column}
    \end{columns}
\end{frame}

%================================================
%	SECTION 4: IMPLEMENTATION - BACKEND
%================================================
\setsectioncontent{Technology stack, web scraping, database integration, and AI features.}
\section{Implementation: Backend}

\subsection{Technology Stack}
\begin{frame}
    \frametitle{Technology Stack}
    \framesubtitle{Backend components}
    
    \begin{columns}
        \begin{column}{0.5\textwidth}
            \begin{itemize}
                \item \textbf{Database}: MySQL
                    \begin{itemize}
                        \item Originally planned PostgreSQL
                        \item Switched to MySQL for compatibility
                    \end{itemize}
                \item \textbf{Backend}: Flask (Python)
                \item \textbf{ORM}: SQLAlchemy
            \end{itemize}
        \end{column}
        \begin{column}{0.5\textwidth}
            \begin{itemize}
                \item \textbf{Web Scraping}: BeautifulSoup
                \item \textbf{Authentication}: JWT + bcrypt
                \item \textbf{AI Translation}: Ollama LLM
                \item \textbf{Data Format}: JSON APIs
            \end{itemize}
        \end{column}
    \end{columns}
    
    \vspace{0.2cm}
    \centering
    \includegraphics[width=0.35\textwidth]{figures/tech_stack.png}
\end{frame}

\subsection{Web Scraping System}
\begin{frame}
    \frametitle{Web Scraping System}
    \framesubtitle{Real-time data acquisition from Drugs.com}
    
    \begin{itemize}
        \item \textbf{DrugInteractionChecker}: Primary scraper class
        \item \textbf{FoodInteractionScraper}: Food/lifestyle interactions
        \item \textbf{DiseaseInteractionScraper}: Disease-related interactions
        \item HTML parsing with BeautifulSoup
        \item Intelligent caching to reduce redundant requests
        \item Brand name $\rightarrow$ Generic name resolution (e.g., Valium $\rightarrow$ Diazepam)
    \end{itemize}
    
    \centering
    \includegraphics[width=0.28\textwidth]{figures/scraping_workflow.png}
\end{frame}

\subsection{Web Scraping Code}
\begin{frame}[fragile]
    \frametitle{Web Scraping System}
    \framesubtitle{[CODE DEMO] scraper.py - DrugInteractionChecker}
    
    \vspace{0.5cm}
    \centering
    \textit{[DEMO: Show scraper.py in IDE]}
    
    \vspace{0.3cm}
    Key methods to highlight:
    \begin{itemize}
        \item \texttt{get\_drug\_interactions()} -- Fetches drug-drug interactions
        \item \texttt{get\_food\_interactions()} -- Fetches food/lifestyle interactions
        \item \texttt{\_get\_generic\_name()} -- Resolves brand names
    \end{itemize}
\end{frame}

\subsection{Database Integration}
\begin{frame}[fragile]
    \frametitle{Database Integration}
    \framesubtitle{SQLAlchemy ORM models}
    
    \begin{lstlisting}[language=Python, basicstyle=\tiny\ttfamily]
class User(Base):
    __tablename__ = 'User'
    
    user_id = Column(Integer, primary_key=True, autoincrement=True)
    username = Column(String(64), nullable=False, unique=True)
    password_hash = Column(String(60), nullable=False)
    email = Column(String(255), nullable=False, unique=True)
    role = Column(Enum('PATIENT', 'DOCTOR'), nullable=False)
    
    # Relationships
    search_history = relationship('SearchHistory', back_populates='user')
    patients = relationship('User', secondary=doctor_patient_table,
                           primaryjoin=(user_id == doctor_patient_table.c.doctor_id),
                           secondaryjoin=(user_id == doctor_patient_table.c.patient_id))
    \end{lstlisting}
    
    \centering
    \textit{[DEMO: Show database.py in IDE]}
\end{frame}

\subsection{API Endpoints}
\begin{frame}
    \frametitle{API Endpoints}
    \framesubtitle{RESTful API design}
    
    \begin{columns}
        \begin{column}{0.5\textwidth}
            \textbf{Authentication}
            \begin{itemize}
                \item POST /auth/register
                \item POST /auth/login
                \item GET /auth/me
            \end{itemize}
            
            \textbf{Drug Search}
            \begin{itemize}
                \item GET /search\_drugs
                \item GET /search\_conditions
            \end{itemize}
        \end{column}
        \begin{column}{0.5\textwidth}
            \textbf{Interactions}
            \begin{itemize}
                \item POST /check\_drug\_interactions
                \item GET /food\_interactions
                \item GET /disease\_interactions
            \end{itemize}
            
            \textbf{Doctor-Patient}
            \begin{itemize}
                \item GET /doctors/patients
                \item POST /patients/request\_doctor
            \end{itemize}
        \end{column}
    \end{columns}
    
    \vspace{0.3cm}
    \centering
    \textit{[DEMO: Show api.py endpoints in IDE]}
\end{frame}

\subsection{AI Translation Feature}
\begin{frame}
    \frametitle{AI Translation Feature}
    \framesubtitle{Ollama LLM integration}
    
    \begin{itemize}
        \item Translates professional medical descriptions to patient-friendly language
        \item Uses locally-deployed Ollama LLM (privacy-preserving)
        \item Caches translations in database for performance
        \item Users can toggle between professional and AI-translated views
    \end{itemize}
    
    \vspace{0.3cm}
    \begin{columns}
        \begin{column}{0.5\textwidth}
            \textbf{Before (Professional)}
            \begin{block}{}
                \footnotesize
                ``Using fluoxetine together with diazePAM may increase side effects such as dizziness, drowsiness, confusion, and difficulty concentrating...''
            \end{block}
        \end{column}
        \begin{column}{0.5\textwidth}
            \textbf{After (AI Translated)}
            \begin{block}{}
                \footnotesize
                ``Taking these two medications together may make you feel more dizzy or sleepy than usual. Be careful when driving...''
            \end{block}
        \end{column}
    \end{columns}
\end{frame}

%================================================
%	SECTION 5: IMPLEMENTATION - FRONTEND
%================================================
\setsectioncontent{User interface design, core features, and doctor-patient views.}
\section{Implementation: Frontend}

\subsection{User Interface Design}
\begin{frame}
    \frametitle{User Interface Design}
    \framesubtitle{Modern, responsive web application}
    
    \begin{itemize}
        \item Built with vanilla HTML, CSS, and JavaScript
        \item No heavy framework overhead
        \item Flask serves static files directly
        \item Responsive design for desktop and mobile
        \item Clean, medical-professional aesthetic
    \end{itemize}
    
    \vspace{0.2cm}
    \centering
    \includegraphics[width=0.55\textwidth]{figures/welcome_page.png}
\end{frame}

\subsection{Core Features}
\begin{frame}
    \frametitle{Core Features}
    \framesubtitle{Drug interaction checking capabilities}
    
    \begin{columns}
        \begin{column}{0.5\textwidth}
            \begin{itemize}
                \item Multi-drug interaction checker (up to 5 drugs)
                \item Autocomplete drug search
                \item Severity-coded results (Major, Moderate, Minor)
                \item Expandable interaction details
            \end{itemize}
        \end{column}
        \begin{column}{0.5\textwidth}
            \begin{itemize}
                \item Food/lifestyle interaction checker
                \item Disease interaction checker
                \item Search history with clickable restoration
                \item AI translation on-demand
            \end{itemize}
        \end{column}
    \end{columns}
    
    \vspace{0.2cm}
    \centering
    \includegraphics[width=0.2\textwidth]{figures/drug_checker.png}
\end{frame}

\subsection{Doctor-Patient Features}
\begin{frame}
    \frametitle{Doctor-Patient Features}
    \framesubtitle{Oversight and monitoring capabilities}
    
    \begin{itemize}
        \item \textbf{Patient Registration}: Select a doctor during signup
        \item \textbf{Doctor Dashboard}: View all assigned patients
        \item \textbf{Search History Access}: Doctors can view patient search history
        \item \textbf{Real-time Tracking}: Recent searches displayed at a glance
        \item \textbf{Patient Control}: Patients can add/remove doctor assignments
    \end{itemize}
    
    \vspace{0.2cm}
    \centering
    \includegraphics[width=0.6\textwidth]{figures/dashboard.png}
\end{frame}

\subsection{Patient Management View}
\begin{frame}
    \frametitle{Patient Management View}
    \framesubtitle{Doctor's patient list interface}
    
    \centering
    \includegraphics[width=0.7\textwidth]{figures/my_patients.png}
\end{frame}

%================================================
%	SECTION 6: LIVE DEMONSTRATION
%================================================
\setsectioncontent{Live walkthrough of system features.}
\section{Live Demonstration}

\subsection{Demo: Registration}
\begin{frame}
    \frametitle{Demo Part 1: User Registration \& Login}
    \framesubtitle{[LIVE DEMO]}
    
    \vspace{1cm}
    \centering
    \Large
    \textbf{Demonstrating:}
    
    \vspace{0.5cm}
    \begin{itemize}
        \item Creating a new patient account
        \item Selecting a doctor from dropdown
        \item Login authentication flow
        \item Dashboard navigation
    \end{itemize}
\end{frame}

\subsection{Demo: Drug Interactions}
\begin{frame}
    \frametitle{Demo Part 2: Drug Interaction Check}
    \framesubtitle{[LIVE DEMO]}
    
    \vspace{1cm}
    \centering
    \Large
    \textbf{Demonstrating:}
    
    \vspace{0.5cm}
    \begin{itemize}
        \item Checking interactions between Prozac and Valium
        \item Viewing severity levels (color-coded)
        \item Expanding interaction details
        \item Browsing through multiple interactions
    \end{itemize}
\end{frame}

\subsection{Demo: AI Translation}
\begin{frame}
    \frametitle{Demo Part 3: AI Translation}
    \framesubtitle{[LIVE DEMO]}
    
    \vspace{1cm}
    \centering
    \Large
    \textbf{Demonstrating:}
    
    \vspace{0.5cm}
    \begin{itemize}
        \item Clicking ``Translate to Patient-Friendly'' button
        \item Animated loading dots during translation
        \item Before/after description comparison
        \item Translation caching behavior
    \end{itemize}
\end{frame}

\subsection{Demo: Food \& Disease}
\begin{frame}
    \frametitle{Demo Part 4: Food \& Disease Interactions}
    \framesubtitle{[LIVE DEMO]}
    
    \vspace{1cm}
    \centering
    \Large
    \textbf{Demonstrating:}
    
    \vspace{0.5cm}
    \begin{itemize}
        \item Checking food/lifestyle interactions for Diazepam
        \item Checking disease interactions
        \item Brand name to generic name resolution
        \item Dashboard navigation between checkers
    \end{itemize}
\end{frame}

\subsection{Demo: Doctor Dashboard}
\begin{frame}
    \frametitle{Demo Part 5: Doctor Dashboard}
    \framesubtitle{[LIVE DEMO]}
    
    \vspace{1cm}
    \centering
    \Large
    \textbf{Demonstrating:}
    
    \vspace{0.5cm}
    \begin{itemize}
        \item Logging in as a doctor account
        \item Viewing list of assigned patients
        \item Seeing patient's recent searches at a glance
        \item Accessing full patient search history
    \end{itemize}
\end{frame}

%================================================
%	SECTION 7: CHALLENGES
%================================================
\setsectioncontent{Technical, design, and UI/UX challenges encountered during development.}
\section{Challenges \& Solutions}

\subsection{Technical Challenges}
\begin{frame}
    \frametitle{Technical Challenges}
    \framesubtitle{Backend and scraping issues}
    
    \begin{itemize}
        \item \textbf{HTML Structure Changes}: Drugs.com format evolved during development
            \begin{itemize}
                \item Solution: Robust parsing with multiple fallback selectors
            \end{itemize}
        \item \textbf{Brand Name Resolution}: Prozac $\neq$ Fluoxetine in URLs
            \begin{itemize}
                \item Solution: Implemented \texttt{\_get\_generic\_name()} lookup
            \end{itemize}
        \item \textbf{Performance}: Initial scraping took 2+ minutes
            \begin{itemize}
                \item Solution: Caching mechanism with \texttt{use\_cache=True}
            \end{itemize}
        \item \textbf{Python Indentation Errors}: Inconsistent formatting
            \begin{itemize}
                \item Solution: Careful code review and syntax checking
            \end{itemize}
    \end{itemize}
\end{frame}

\subsection{Design Challenges}
\begin{frame}
    \frametitle{Design Challenges}
    \framesubtitle{Architecture and schema evolution}
    
    \begin{itemize}
        \item \textbf{Schema Evolution}: Added FoodInteraction and DiseaseInteraction tables mid-development
        \item \textbf{Search History}: Implementing clickable restoration required storing full JSON results
        \item \textbf{Doctor-Patient Flow}: Original design had doctors adding patients
            \begin{itemize}
                \item Reversed to: Patients request doctor oversight
            \end{itemize}
        \item \textbf{File Protocol Issues}: Direct file:// serving broke navigation
            \begin{itemize}
                \item Solution: Flask serves all static files via HTTP
            \end{itemize}
    \end{itemize}
\end{frame}

\subsection{UI/UX Challenges}
\begin{frame}
    \frametitle{UI/UX Challenges}
    \framesubtitle{Frontend polish and user experience}
    
    \begin{itemize}
        \item \textbf{Logo Display}: SVG logo appeared as white rectangle on some pages
            \begin{itemize}
                \item Solution: Removed brightness/invert CSS filters
            \end{itemize}
        \item \textbf{Dropdown State}: Cards collapsed after AI translation
            \begin{itemize}
                \item Solution: Re-expand card after render using index tracking
            \end{itemize}
        \item \textbf{Multi-Selector}: Comma-separated input was error-prone
            \begin{itemize}
                \item Solution: Tag-based multi-selector with autocomplete
            \end{itemize}
        \item \textbf{Loading States}: No feedback during long operations
            \begin{itemize}
                \item Solution: Animated loading dots CSS animation
            \end{itemize}
    \end{itemize}
\end{frame}

%================================================
%	SECTION 8: BETA I ACHIEVEMENTS
%================================================
\setsectioncontent{First working system achievements and metrics.}
\section{Beta I: First Working System}

\subsection{Achievements}
\begin{frame}
    \frametitle{Beta I Achievements}
    \framesubtitle{First working version milestones}
    
    \begin{columns}
        \begin{column}{0.5\textwidth}
            \begin{itemize}
                \item Fully functional MySQL database with 8+ tables
                \item Complete web scraping pipeline
                \item Working AI translation feature
                \item JWT-based authentication system
                \item Role-based access control
            \end{itemize}
        \end{column}
        \begin{column}{0.5\textwidth}
            \begin{itemize}
                \item Doctor-patient relationship management
                \item Three interaction checkers (Drug, Food, Disease)
                \item Search history with restoration
                \item Responsive web interface
                \item Real-time data from Drugs.com
            \end{itemize}
        \end{column}
    \end{columns}
    
\end{frame}

\subsection{Data Import}
\begin{frame}
    \frametitle{Data Import Process}
    \framesubtitle{Populating the database with drugs and conditions}
    
    \begin{columns}
        \begin{column}{0.5\textwidth}
            \centering
            \includegraphics[width=0.9\textwidth]{figures/import_data_1.png}
        \end{column}
        \begin{column}{0.5\textwidth}
            \begin{itemize}
                \item Imported \textbf{2,123 conditions} from conditions.json
                \item Imported \textbf{15,773 drugs} from drugs.json
                \item Batch processing with progress feedback
                \item Data sourced from Drugs.com API
            \end{itemize}
        \end{column}
    \end{columns}
\end{frame}

\subsection{Metrics}
\begin{frame}
    \frametitle{Beta I Metrics}
    \framesubtitle{Project statistics}
    
    \begin{columns}
        \begin{column}{0.5\textwidth}
            \textbf{Codebase Size}
            \begin{itemize}
                \item \textbf{api.py}: 1,115 lines
                \item \textbf{scraper.py}: 927 lines
                \item \textbf{database.py}: 351 lines
                \item \textbf{auth.py}: 303 lines
                \item Frontend: 1,500+ lines (HTML/CSS/JS)
            \end{itemize}
        \end{column}
        \begin{column}{0.5\textwidth}
            \textbf{Data Coverage}
            \begin{itemize}
                \item \textbf{15,775} drugs in database
                \item \textbf{2,126} medical conditions
                \item Real-time access to all Drugs.com interactions
                \item Support for food and disease interactions
            \end{itemize}
        \end{column}
    \end{columns}
\end{frame}

%================================================
%	SECTION 9: FUTURE WORK
%================================================
\setsectioncontent{Planned enhancements and technical improvements.}
\section{Future Work}

\subsection{Planned Enhancements}
\begin{frame}
    \frametitle{Planned Enhancements}
    \framesubtitle{Feature roadmap}
    
    \begin{itemize}
        \item \textbf{Doctor-Patient Communication}: Direct messaging system
        \item \textbf{Medication Reminders}: Scheduled notifications
        \item \textbf{Prescription Management}: Track current prescriptions
        \item \textbf{Drug Allergy Tracking}: Alert on known allergens
        \item \textbf{Enhanced Patient Profiles}: Medical history, demographics
        \item \textbf{Doctor Notes}: Add annotations to patient interactions
    \end{itemize}
\end{frame}

\subsection{Technical Improvements}
\begin{frame}
    \frametitle{Technical Improvements}
    \framesubtitle{Infrastructure and performance}
    
    \begin{itemize}
        \item \textbf{Background Job Queue}: Async scraping with Celery
        \item \textbf{Full-Text Search}: Elasticsearch integration
        \item \textbf{Mobile Application}: React Native or Flutter app
        \item \textbf{PDF Reports}: Export search history as reports
        \item \textbf{Pharmacy Integration}: Connect with pharmacy systems
        \item \textbf{Analytics Dashboard}: Patient adherence tracking for doctors
    \end{itemize}
\end{frame}

%================================================
%	CONCLUSION
%================================================
\subsection{Conclusion}
\begin{frame}
    \frametitle{Conclusion}
    \framesubtitle{Project summary}
    
    \begin{itemize}
        \item Successfully built an end-to-end drug interaction checking system
        \item Implements real-time web scraping with intelligent caching
        \item AI-powered translation makes medical information accessible
        \item Doctor-patient features enable professional oversight
        \item Personal goal achieved: Building a tool I would have wanted during treatment
        \item Potential for real-world impact in patient medication safety
    \end{itemize}
    
    \vspace{0.5cm}
    \centering
    \Large
    \textbf{Thank you!}
\end{frame}

%================================================
%	FINAL DEMO
%================================================
\begin{frame}
    \frametitle{Complete System Demonstration}
    \framesubtitle{Full walkthrough}
    
    \vspace{1cm}
    \centering
    \Huge
    \textbf{[LIVE DEMO]}
    
    \vspace{0.5cm}
    \Large
    Complete walkthrough of PharmaCheck
    
    \vspace{0.5cm}
    \normalsize
    \url{http://localhost:5000}
\end{frame}

%================================================
%	REFERENCES
%================================================
\beamerdefaultoverlayspecification{}
\begin{frame}[plain]
    \frametitle{References}
    \printbibliography
\end{frame}

\conclusionpage
\end{document}
